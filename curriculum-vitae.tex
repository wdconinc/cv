%%---------------------------------------------------------------------------
%
%% Notes:
%
% * To create a new page use
%   \newpage \opening
%
% * res.cls includes an \address{} command which can be used up to twice,
%   but my address is too long for the format it uses.
%
% * Alternate documentclass command:
%   \documentclass[margin,line,11pt,final]{res}
%
% * When publication/presentation list gets longer, divide into subsections
%   by year like this:
%   \section{\bf\small\hspace{8mm}2005}
%
%%----------------------------------------------------------------------------

\documentclass[overlapped,line,final,11pt,letterpaper]{res}

\usepackage{multirow,hyperref}

\hypersetup{
  colorlinks,urlcolor=black,
  pdftitle={Wouter Deconinck: Curriculum Vitae},
  pdfauthor={Wouter Deconinck},
  pdfcreator={\LaTeX\ with package \flqq hyperref\frqq},
  pdfsubject={Curriculum Vitae},
  pdfkeywords={exotic baryons hermes parity violation electron scattering qweak}
}

%%===========================================================================%%

\begin{document}

\name{Wouter Deconinck}
\address{\href{mailto:Wouter Deconinck <wdeconinck@wm.edu>}{\texttt{wdeconinck@wm.edu}}}

\begin{resume}
{\bf Wouter Deconinck} \\
{\bf Assistant Professor of Physics} \\
{\bf College of William \& Mary} \\
Department of Physics \\
P.O. Box 8795         \\
Williamsburg, VA 2387 \\
Phone: (757) 221-3539 \\

%---------------------------------------------------------------------------

\section{\bf Education}

Ph.D. in Physics, April 2008 (Professor Wolfgang Lorenzon)\\
``The Search for Exotic Baryons at the \textsc{Hermes} Experiment'' \\
University of Michigan, Ann Arbor, Michigan
\vspace{0.5\baselineskip} \\
M.S. in Engineering Physics, with great distinction, July 2003 (Professor Dirk Aeyels) \\
University of Gent, Belgium

%---------------------------------------------------------------------------

\section{\bf Appointments}

\begin{format}
\employer{l}\location{r}\\
\title{l}\dates{r}\\
\body\\
\end{format}

\employer{\bf College of William \& Mary}
\title{Assistant Professor}
\location{Williamsburg, Virginia}
\dates{May 2010--present}
\begin{position}
  Main focus on Compton polarimetry and track reconstruction for the $Q_{Weak}$ experiment at Jefferson Lab, resulting in the first determination of the proton's weak charge in Fall 2013.  Simulations and design for the next generation of parity-violating electron scattering experiments at Jefferson Lab (\textsc{Moller} and \textsc{SoLID}) and the University of Mainz (P2).  Development of atomic hydrogen polarimetry to reach sub-percent precision in electron polarimetry in collaboration with the University of Mainz.  Teaching of intro- to upper-level undergraduate courses, electronics laboratory, and graduate level courses.
\end{position}

\employer{\bf Massachusetts Institute of Technology}
\title{Postdoctoral Associate}
\location{Jefferson Lab, Virginia}
\dates{June 2008--April 2010}
\begin{position}
  Professors Aron~Bernstein and Stanley~Kowalski
  \\
  Responsible for the design and construction of a photon detector for the Compton polarimeter in Hall C at Jefferson Lab.  Coordinator of the track reconstruction and simulation efforts in parity-violating elastic electron scattering on a hydrogen target to determine the weak charge of the proton in the $Q_{Weak}$ experiment at Jefferson Lab.  Spokesperson on a proposed measurement of neutral pion photoproduction asymmetries in the threshold region at the A2 experiment in Mainz, Germany.
\end{position}

\employer{\bf Deutsches Elektronen Synchrotron}
\title{Research Fellow}
\location{\textsc{Hermes}, DESY, Germany}
\dates{January--May 2008}
\begin{position}
  Professor Wolf-Dieter~Nowak
  \\
  Development of a magnetic field tracking framework for secondary baryons with displaced vertices.  Preparation of a publication with results on exotic baryons in data collected at the \textsc{Hermes} experiment.
\end{position}

%---------------------------------------------------------------------------

\begin{itemize}
\section{\bf Current and Past Funding}
 \item \textbf{National Science Foundation}, PHY-1405857, August 2014--July 2017, ``Precision Studies of the Standard Model using Parity-Violating Electron Scattering'' (co-PI with David Armstrong), awarded amount: \$900,000.
 \item \textbf{National Science Foundation}, PHY-1206053, May 2012--April 2015, ``Precision Electroweak Measurements using Parity-Violating Polarized Electron Scattering'' (sole PI), awarded mount: \$300,000
 \item \textbf{National Science Foundation}, PHY-1359364, March 2014--Feburary 2017, ``REU Site: Physics Research in America's Historic Triangle'' (PI with one co-PI), awarded amount: \$309,000
  \item \textbf{National Science Foundation}, PHY-1156997, May 2011--April 2014, ``REU Site: Physics Research Experiences for Undergraduates'' (PI with one co-PI), awarded amount: \$180,000
 \item \textbf{Institute for Nuclear Theory}, 2013--present, ``National Nuclear Physics Summer School 2014'' (PI with several co-PIs), awarded amount: \$60,400
 \item \textbf{Brookhaven National Lab}, Electron-Ion Collider R\&D (co-PI with Dipangkar Dutta, Mississippi State University), awarded amount: \$15,000
 \item \textbf{Jefferson Science Associates}, 2011--present, ``Promising Young Scientist: Career Development for Postdoctoral Researchers at Jefferson Lab'' (PI with several co-PIs), awarded amount: $\sim$\$6,000/year
\end{itemize}

%---------------------------------------------------------------------------

\section{\bf Awards and Fellowships}

Margaret and Herman Sokol International Research Fellowship in the Sciences, 2005 \\
Rackham Graduate School, University of Michigan
\vspace{0.5\baselineskip} \\
Wirt and Mary Cornwell Award, 2005 \\
Physics Department, University of Michigan
\vspace{0.5\baselineskip} \\
Distinguished Dissertation Award (honorable mention), 2008 \\
Rackham Graduate School, University of Michigan
\vspace{0.5\baselineskip} \\
Fellowship in Experimental Particle Physics, 2008 \\
Deutsches Elektronen-Synchrotron (DESY), Germany

%---------------------------------------------------------------------------

\section{\bf Teaching and Mentoring}
\vspace{0.5\baselineskip}

College of William \& Mary, Williamsburg, Virginia
\begin{itemize}
 \item Physics for Life-Sciences, Fall 2010 (recitations), Spring 2013 (recitations), Fall 2014.
 \item Introduction to Mathematical Physics, Spring 2011
 \item Media Production for Scientists, Fall 2011 (acted as scientific adviser)
 \item Introductory Physics (recitations), Spring 2012
 \item Analog Electronics (lecture and laboratory), Spring 2013/2015
 \item Science Writing (independent study), Fall 2012
 \item Nuclear and Particle Physics, Fall 2012
 \item Graduate Classical Mechanics, Fall 2011/2012/2013
\end{itemize}

Mentoring activities at the College of William \& Mary
\begin{itemize}
 \item PI and coordinator of the long-running physics REU program at College of William \& Mary since 2011
 \item Ph.D. students supervised: 5; Juan Carlos Cornejo, Valerie Gray, Kurtis Bartlett, Darren Getts, Melissa Beebe
 \item Undergraduate students supervised: 4 honors thesis students, 2 senior thesis students, 7 REU summer students, 1 high school student, and 6 students for regular academic credit during the semester
\end{itemize}

%---------------------------------------------------------------------------

\section{\bf Service and Synergistic Activities}
\vspace{0.5\baselineskip}

College of William \& Mary, Williamsburg, Virginia
\begin{itemize}
 \item Graduate admissions committee, Fall 2010--present
 \item Computing committee, Fall 2011--Summer 2012
 \item External relations committee, Fall 2012--present
 \item Pre-major adviser, Fall 2011--present
 \item LGBT Climate Working Group, reporting to the president, Summer 2012
\end{itemize}

Jefferson Lab
\begin{itemize}
 \item Software maintainer, data quality and statistics focus group leader, and simulation convener in the Qweak collaboration
 \item Coordinating committee member of the Jefferson Lab Hall A collaboration, Spring 2012--2014
\end{itemize}

Nuclear Physics Community
\begin{itemize}
 \item Coordinator for electron beam polarimetry at the Electron-Ion Collider, Fall 2010--present
 \item Co-author of ``Science Requirements and Conceptual Design for a Polarized Medium Energy Electron-Ion Collider at Jefferson Lab,'' a white paper on the EIC design at Jefferson Lab
 \item Convener source/polarimetry working group of EIC14 conference at Jefferson Lab
 \item Organizer of the National Nuclear Physics Summer School 2014 at the College of William \& Mary; National Nuclear Physics Summer School Steering Committee member, 2015--present
 \item Reviewer for Nuclear Instruments and Methods, National Science Foundation
\end{itemize}

American Physical Society (APS) and American Association of Physics Teachers (AAPT)
\begin{itemize}
 \item Organizing member of LGBT+Physicists, an APS-supported organization to improve the climate for gender and sexual minorities in physics; co-organizer of invited session at APS March 2012 meeting (with APS Committee on Minorities and Status of Women in Physics) on ``Sexual and Gender Diversity in Physics''; co-organizer of discussion sessions at APS April 2011 and March 2013 meetings
 \item Invited panel talk at the APS/AAPT Physics Department Chairs Conference on ``Developing an Inclusive Diversity Climate,'' June 2012
 \item Member of the American Physical Society's ad-hoc committee on LGBT+ issues in physics, September 2014--present.
\end{itemize}

Public Outreach
\begin{itemize}
 \item Founder and organizer of ``Saturday Morning Physics,'' a monthly public lecture series at the College of William \& Mary, Spring 2012--present
 \item Co-founder and co-organizer of PhysicsFest, the open house of the Department of Physics, College of William \& Mary, Fall 2011/2012/2013
\end{itemize}

%------------------------------------------------------------------------------
%\pagebreak
%------------------------------------------------------------------------------

\section{\bf Presentations}

\begin{enumerate}
\section{\bf\small\quad Conferences}
\item ``Parity-violating and parity-conserving asymmetries in electron-proton and electron-nucleus scattering at the Qweak experiment,'' Southeastern Section of the APS (SESAPS), Columbia, South Carolina, November 2014.
\item ``Atomic Hydrogen Polarimetry for Precision Electroweak Experiments,'' APS Division of Nuclear Physics Fall 2012 Meeting, Newport Beach, California, October 2012
\item ``Recent Results and Future Plans from the A4 Experiment,'' APS April 2011 meeting, Washington, DC, May 2011
\item ``Precision Compton Polarimetry for the Qweak Experiment at Jefferson Lab,'' 4th International Workshop ``From Parity Violation to Hadronic Structure and more...'' (PAVI), June 2009, Bar Harbor, Maine
\item ``Experimental Techniques in Hadron Spectroscopy,'' Gordon Research Conference on Photonuclear Reactions, August 2008, Tilton, New Hampshire
\item ``Hybrid Electron-Compton Polarimeter with Online Self-Calibration,'' Precision Electron Beam Polarimetry for the Electron Ion Collider, August 2007, Ann Arbor, Michigan
\item ``Search for Exotic Baryons at \textsc{Hermes},'' Quarks and Nuclear Physics, June 2006, Madrid, Spain
\item ``Search for Exotic Baryons at \textsc{Hermes},'' American Physical Society, April Meeting 2005, Tampa, Florida
\item ``Search for Exotic Baryons at \textsc{Hermes},'' American Physical Society, April Meeting 2004, Denver, Colorado
\end{enumerate}

\begin{enumerate}
\section{\bf\small\quad Invited Talks}
\item ``Parity-Conserving and Parity-Violating Asymmetries in Electron-Proton and Electron-Nucleus in the Qweak Experiment,'' Winter Workshop on Nuclear Dynamics, Keystone, Colorado, January 2015.
\item ``Queer Geeks: Creating an LGBT+ Inclusive Climate in the Physical Sciences,'' Gender, Sexuality and Women Sudies brown bag lunch seminar, William \& Mary, Williamsburg, Virginia, September 2014.
\item ``Sub-Percent Electron Polarimetry for the EIC,'' Electron-Ion Collider 2014 accelerator workshop at JLab, May 2014.
\item ``The Qweak Experiment: The First Determination of the Proton’s Weak Charge,'' physics colloquium at Kenyon College, March 2014.
\item ``The Qweak Experiment: The First Determination of the Proton’s Weak Charge,'' nuclear physics seminar at BNL, February 2014.
\item ``The Qweak Experiment: The First Determination of the Proton’s Weak Charge,'' nuclear physics colloquium at MIT, February 2014.
\item ``The Qweak Experiment: The First Determination of the Proton’s Weak Charge,'' astro/nuclear/high energy physics seminar at University of Michigan, January 2014.
\item ``The First Determination of the Weak Charge of the Proton,'' Laboratory for Nuclear Science Seminar, Massachusetts Institute of Technology, Cambridge, Massachusetts, September 2013
\item ``Precision Measurements at Jefferson Lab: Testing the Standard Model and Exploring Beyond,'' LNF Mini-workshop on Jefferson Lab at 12 GeV, Frascati, Italy, December 2012
\item ``The $Q_{Weak}$ Experiment: A Search for New Physics at the TeV Scale,'' Jefferson Lab User's Group Meeting, Newport News, June 2012
\item ``The $Q_{Weak}$ Experiment: a Search for New Physics at the TeV Scale,'' DSPIN-2011, Dubna, Russia
\item ``The $Q_{Weak}$ Experiment: A Search for Physics at the TeV Scale,'' Laboratory for Nuclear Science Seminar, Massachusetts Institute of Technology, Cambridge, Massachusetts, November 2009
\item ``Pentaquarks: Much Ado About Nothing?'' Laboratory for Nuclear Science Seminar, Massachusetts Institute of Technology, Cambridge, Massachusetts, December 2007
\item ``Pentaquarks: Much Ado About Nothing?'' Nuclear Physics Seminar, University of Virginia, Charlottesville, Virginia, October 2007
\item ``Pentaquarks: Exotic Species Almost Extinct?'' Graduate Student Mini-Colloquium, University of Michigan, January 2007
\item ``Pentaquarks: Fact or Myth?'' Nuclear Physics Seminar, University of Gent, Belgium, November 2006
\item ``Search for Exotic Baryons at \textsc{Hermes},'' National Nuclear Physics Summer School, Indiana University, Bloomington, Indiana, July 2006
\item ``Exotic Baryons at \textsc{Hermes},'' Graduate Student Mini-Colloquium, University of Michigan, December 2005
\item ``Exotic Baryons at \textsc{Hermes},'' Graduate Student Seminar, DESY, Germany, July 2005
\item ``Analysis of Exotic Baryons at \textsc{Hermes},'' Spin Physics Seminar, University of Michigan, November 2004
\item ``Search for Exotic Baryons at \textsc{Hermes},'' Hampton University Graduate School Summer School 2004, Jefferson Lab, Virginia, June 2004

\end{enumerate}

%%---------------------------------------------------------------------------%%

\end{resume}

\end{document}

%%===========================================================================%%
